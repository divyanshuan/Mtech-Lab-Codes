\documentclass[14pt]{article}
\usepackage[a4paper, margin=2.54cm, portrait]{geometry}
\usepackage{amsmath}
\usepackage{graphicx} % Required for inserting images
\usepackage{tikz}
\usepackage{relsize}
\usepackage{pgfplots}
\usepgfplotslibrary{external}
\tikzexternalize

\begin{document}

\begin{center}
{\LARGE
    \textbf{Practical Lab}\\
    \textbf{(CS 614)}\\
    \includegraphics[height=5cm, width=5cm]{logo.png}\\
    Name: Divyanshu Verma\\ 
    Roll Number: 24MCS022 \\ 
    Subject: Cloud Computing \\ 
    Subject Code: CS 614 \\ 
    Class: M.Tech CSE Ist \\ 
    Date: 3 December 2024 \\ 
}
\end{center}

\newpage

\section{Abstract:}
Cloud computing has revolutionized how individuals and organizations manage, store, and process data. It offers scalable resources over the internet, enabling efficient computing services on-demand. This article provides an overview of cloud computing, discussing its architecture, deployment models, and key benefits. Furthermore, it examines security concerns and presents solutions for safe cloud usage, which is crucial for organizations migrating their infrastructure to the cloud.

\section{Introduction:}
Cloud computing is a model for enabling ubiquitous, on-demand access to a shared pool of configurable computing resources (e.g., networks, servers, storage, applications, and services) that can be rapidly provisioned and released with minimal management effort or service provider interaction. The rise of cloud computing has transformed traditional IT infrastructures, allowing companies to offload hardware costs and management to cloud providers.

Cloud computing consists of three main service models:

\begin{itemize}
    \item \textbf{Infrastructure as a Service (IaaS):} Provides virtualized computing resources over the internet. Users can rent servers, storage, and networking infrastructure.
    \item \textbf{Platform as a Service (PaaS):} Provides a platform allowing customers to develop, run, and manage applications without dealing with the underlying infrastructure.
    \item \textbf{Software as a Service (SaaS):} Offers software applications over the internet on a subscription basis, eliminating the need for local installation and maintenance.
\end{itemize}

\section{Deployment Models:}
Cloud computing can be deployed using various models depending on the organization's needs:

\begin{itemize}
    \item \textbf{Public Cloud:} The services are delivered over the public internet, accessible to anyone who wants to use or purchase them. They are often cost-effective due to shared infrastructure.
    \item \textbf{Private Cloud:} Cloud infrastructure operated solely for a single organization. It can be managed internally or by a third party, offering more control over data security and privacy.
    \item \textbf{Hybrid Cloud:} A combination of public and private clouds, where data and applications can move between them for greater flexibility and optimization of existing infrastructure.
\end{itemize}

\section{Benefits of Cloud Computing:}
Cloud computing offers several advantages over traditional IT systems:

\begin{itemize}
    \item \textbf{Scalability:} Resources can be scaled up or down based on demand.
    \item \textbf{Cost Efficiency:} Pay-as-you-go model eliminates upfront capital expenditure.
    \item \textbf{Accessibility:} Cloud services are accessible from anywhere with an internet connection.
    \item \textbf{Disaster Recovery:} Cloud providers often include backup and recovery solutions, ensuring data is protected from hardware failure or natural disasters.
\end{itemize}

\section{Security Concerns and Solutions:}
While cloud computing offers numerous benefits, it also introduces security challenges. The main concerns include data breaches, loss of control over data, and insecure APIs. To mitigate these risks, organizations should consider the following security practices:

\begin{itemize}
    \item \textbf{Data Encryption:} Encrypting data both at rest and in transit ensures unauthorized users cannot access sensitive information.
    \item \textbf{Identity and Access Management (IAM):} Implementing strict access controls and authentication mechanisms limits who can access cloud resources.
    \item \textbf{Compliance:} Organizations should ensure their cloud providers comply with relevant regulations, such as GDPR, HIPAA, or industry-specific standards.
\end{itemize}

\section{Conclusion:}
Cloud computing has emerged as a vital technology, offering flexibility, scalability, and cost-efficiency for businesses and individuals. Despite its numerous advantages, security remains a critical concern. By adopting appropriate measures like encryption and identity management, organizations can ensure that their data and applications are secure in the cloud environment. As the technology continues to evolve, its role in digital transformation will only expand, bringing new opportunities and challenges in the future.

\end{document}
